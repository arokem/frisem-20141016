% Talk about software engineering practices for scientific workflows
% Ariel Rokem
% License: CC(http://creativecommons.org/licenses/by-nc-sa/3.0/)

\documentclass{beamer}

\usepackage{ulem}
\usepackage{listings,bera}
\usepackage{graphics}
\usepackage{media9}

\setbeamercovered{transparent}
\usetheme{CambridgeUS}
\usecolortheme{beaver}
\useinnertheme{rectangles}

\definecolor{fore}{RGB}{0,20,30}
\definecolor{back}{RGB}{255,255,255}
\definecolor{title}{RGB}{140,21,21}

\setbeamercolor{titlelike}{fg=title}
\setbeamercolor{normal text}{fg=fore,bg=back}
\definecolor{keywords}{RGB}{255,0,90}
\definecolor{comments}{RGB}{60,179,113}
\definecolor{strings}{RGB}{120,120,0}

\lstset{language=Python,
keywordstyle=\color{keywords},
commentstyle=\color{comments}\emph,
stringstyle=\color{strings}}

\title[Practical software engineering]{Practical software engineering }
\subtitle{for psychologists}

\author[Ariel Rokem]
{Ariel Rokem}
\date{October 17, 2014}
\institute[Stanford University]
{Stanford University}

% Put in stanford and vista logos:
\pgfdeclareimage[height=1.5cm]{stanford-logo}{figures/stanford_logo}
\setbeamertemplate{itemize}
  
\begin{document}

%Title page:
\begin{frame}
  \titlepage
  \pgfuseimage{stanford-logo}

\end{frame}

%\begin{frame}
%\frametitle{"Dammit Jim, I'm a psychologist, not a programmer!''}
%\begin{itemize}
%\pause
%\item 
% Humans are good at many things
%\pause
%\item 
% Computers are good at other things
%\pause
%\item
% Automate everything you can
%\pause
%\item
% Now, we've settled that, how can we make sure we do it well? 
%\end{itemize}
%\end{frame}

\begin{frame}
\frametitle{Make your life easier}
\pause
And your results easier to reproduce
\end{frame}

\begin{frame}
\pause
Reproducibility is collaborating with people you don't know
\pause 
...including yourself, next
week. \footnotemark[1]\footnotetext[1]{\href{https://twitter.com/SoftwareSaved/status/519909961414242304}{Philip
  Stark}}

\end{frame}

\begin{frame}
\frametitle{3 practices you can adopt}
\begin{itemize}
\pause
\item
Version control
\pause
\item
Testing
\pause
\item
Code review 
\end{itemize}
\end{frame}

\begin{frame}
\frametitle{3 practices you can adopt}
\begin{itemize}
\item
\emph{Version control}
\item
Testing
\item
Code review 
\end{itemize}
\end{frame}

\begin{frame}
\frametitle{Version control}
Who uses version control?
\end{frame}

\begin{frame}
\frametitle{We all do}
 \includegraphics[height=5.7cm]{figures/vcs_for_science.png}
\end{frame}

%\begin{frame}
%\frametitle{Wouldn't you rather be more systematic?}
%There are many Version Control Systems
%\begin{itemize}
%\pause
%\item
%SVN
%\pause
%\item
%CVS 
%\pause
%\item
%GIT
%\end{itemize}
%\end{frame}


%\begin{frame}
%\frametitle{Git}
%Created by Linus Torvalds
%\begin{itemize}
%\pause
%\item
%Distributed
%\pause
%\item
%Fast 
%\pause
%\item
%In very common use
%\end{itemize}
%\end{frame}

\begin{frame}
\frametitle{What it does}
Git keeps snapshots of your file system
\includegraphics[height=2cm]{figures/git1.pdf}
\end{frame}

\begin{frame}
\frametitle{What it does}
You can revert back to a previous version
\includegraphics[height=2cm]{figures/git1.pdf}
\end{frame}


\begin{frame}
\frametitle{Advantages}
Using a proper version control system promotes:
\begin{itemize}
\pause
\item
Reproducibility : logging of every step
\pause
\item
Peace of mind : a robust backup system
\pause
\item
Flexibility : zero-cost branching
\pause
\item
Collaboration : synchronization across multiple computers/people
\end{itemize}
\pause
This is preferable to a system like Dropbox, because it gives you fine-grained
control over what and when you want to version
\end{frame}


\begin{frame}
\frametitle{Easy to branch}
This is a way to try out new ideas
\includegraphics[height=4cm]{figures/git2.pdf}
\pause
\\
Without worrying you might break something
\end{frame}

%\begin{frame}
%\frametitle{An easy backup system}
%With annotation built in
%\end{frame}
%\begin{frame}
%The modern lab notebook?
%\end{frame}


\begin{frame}
\frametitle{Collaboration is easy}
\begin{itemize}
\pause
\item
Github 
\pause
\item
\href{https://education.github.com/}{Github for students}
\end{itemize}
\end{frame}


\begin{frame}
\frametitle{Version everything!}
\begin{itemize}
\pause
\item
Code
\pause
\item
Papers
\pause
\item
Data (of reasonable size)
\pause 
\item
Even presentations
\pause
\item
\href{https://github.com/arokem/frisem-20141016}{This talk}
\end{itemize}
\end{frame}


\begin{frame}
\frametitle{3 practices you can adopt}
\begin{itemize}
\item
Version control
\item
\emph{Testing}
\item
Code review 
\end{itemize}
\end{frame}


\begin{frame}
\frametitle{Automated testing}
\begin{itemize}
\pause
\item
How do you know your code does what it's supposed to do?
\pause
\item
How do you prevent bugs from recurring?
\item
\pause 
How do you build new ideas, without breaking old ones?
\end{itemize}
\end{frame}

\begin{frame}[fragile]
\frametitle{Assert simple test cases}
\begin{lstlisting}
def circle_diameter(radius):
  return(2 * pi * radius)
\end{lstlisting}
\end{frame}

\begin{frame}[fragile]
\frametitle{Assert simple test cases}
\begin{lstlisting}

def test_circle_diameter(radius):
    assert circle_diameter(2) == 2 * pi

\end{lstlisting}
\end{frame}

\begin{frame}[fragile]
\frametitle{Assert simple test cases}
If it gets those wrong -- 
\\ 
it certainly will not do well with more complicated things!
\end{frame}


\begin{frame}[fragile]
\frametitle{Assert your old results}

Regression testing

\end{frame}


\begin{frame}[fragile]
\frametitle{Assert fixes to bugs}

Those tend to creep back in somehow 

\end{frame}



\begin{frame}
\frametitle{The benefits of tests}
\begin{itemize}
\pause
\item
Validation
\pause
\item
Avoid regressions
\pause
\item
Documentation
\pause
\item
Supports improvements - start with a simple/inefficient implementation and refactor
to make it efficient/complex
\end{itemize}
\end{frame}

%\begin{frame}
%\frametitle{The benefits of tests}
%Don't let this happen to you
%\\
%\includegraphics[height=6cm]{figures/lack_of_tests.jpg}
%\end{frame}

\begin{frame}
\frametitle{Test-driven programming}
Consider writing your tests before you write the code
\end{frame}


\begin{frame}
\frametitle{Continuous integration}
Have your version control system run your tests for you
\href{https://travis-ci.org}{Travis}
\end{frame}

\begin{frame}
\frametitle{3 practices you can adopt}
\begin{itemize}
\item
Version control
\item
Testing
\item
\emph{Code review}
\end{itemize}
\end{frame}



\begin{frame}
\frametitle{Code review
  \footnotemark[1]\footnotetext[1]{\href{http://arxiv.org/abs/1311.2412}{Petre,
    Wilson
    (2013)}} \footnotemark[2]\footnotetext[2]{\href{http://arxiv.org/abs/1311.2412}{Petre,
    Wilson
    (2014)}}}
\begin{itemize}
\pause
\item
Rigor
\pause
\item
Reusability
\pause
\item
Knowledge transfer
\end{itemize}
\end{frame}

\begin{frame}
\frametitle{I know my code best!}
\pause
\includegraphics[height=6cm]{figures/two_heads_better_than_one.png}
\end{frame}

\begin{frame}
\frametitle{How to make review work (author)}
\begin{itemize}
\pause
\item 
 Digestible, coherent piece of code
\pause
\item
Not too large
\end{itemize}
\end{frame}


\begin{frame}
\frametitle{How to make review work (author)}
\includegraphics[height=6cm]{figures/small_code_review.png}
\end{frame}


\begin{frame}
\frametitle{How to make review work (author)}
\begin{itemize}
\item 
 Digestible, coherent piece of code
\item
Not too large
\pause
\item
 Explain in advance what you are trying to achieve
\pause
\item 
 Write tests
\pause
\item
 Readability counts!\footnotemark[1]
\end{itemize}
\footnotetext[1]{\href{http://legacy.python.org/dev/peps/pep-0020/}{The zen of python}}
\end{frame}

\begin{frame}
\frametitle{How to make review work (reviewer)}
\begin{itemize}
\pause
\item 
Have clear standards
\pause 
\item 
Don't accept broken windows and technical debt
\pause
\item 
Don't forget to be positive
\pause
\item 
Readability counts! 
\end{itemize}
\end{frame}

\begin{frame}
\frametitle{Take breaks}
\includegraphics[height=6cm]{figures/review_errors}
\end{frame}


\begin{frame}
\frametitle{Modes of code review}
\begin{itemize}
\pause
\item
Asynchronous 
\pause
\item
Synchronous 
\end{itemize}
\end{frame}

\begin{frame}
\frametitle{Asynchronous: github pull requests}
\begin{itemize}
\pause
\item
Author creates a branch with proposed changes 
\pause
\item
Author submits Pull Request through the web interface
\pause
\item
Reviewer reads and comments
\pause
\item
Author revises code
\pause
\item
Et cetera
\end{itemize}
\end{frame}

\begin{frame}
\frametitle{Synchronous: journal club/lab meeting for code}
\begin{itemize}
\pause
\item
Rob Knight, CU Boulder \footnotemark[1]
\footnotetext[1]{\href{http://fperez.org/py4science/code_reviews.html}{h/t Fernando
  Perez}}
\pause
\item
Code is sent in advance (through github?)
\pause
\item 
Project on a screen and fire away!
\end{itemize}
\end{frame}

\begin{frame}
\frametitle{When in doubt}
\includegraphics[height=5.7cm]{figures/review_meetings.png}
\end{frame}

\begin{frame}
\frametitle{Final thoughts}
\begin{itemize}
\pause
\item
Don't let the perfect be the enemy of the good
\\
``Imperfect tests, run frequently, are much better than perfect tests that are
never written at all''\footnotemark[1]
\footnotetext[1]{\href{http://www.martinfowler.com/articles/continuousIntegration.html}{Martin Fowler}}
\pause
\item
Should this be part of the curriculum?
\pause
\item
There are a lot of resources out there
\end{itemize}
\end{frame}

\end{document}
